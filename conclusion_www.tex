\section{Conclusion and future work}

In this paper, we propose using the massive collections of
user-generated photos uploaded to social sharing websites as a source
of observational evidence about the world, and in particular as a way
of estimating the presence of ecological phenomena. As a first step
towards this long-term goal, we used a collection of 150 million
geo-tagged, timestamped photos from Flickr to estimate snow 
cover and greenery, and compared these estimates to fine-grained
ground truth collected by earth-observing satellites and ground stations. We compared
several techniques for performing the estimation from noisy, biased
data, including simple voting mechanisms and a Bayesian likelihood
ratio. We also tested several possible improvements to these basic
%%hp cr: removed "multi-language tag sets" as we did not report it in
%%this version
methods, including using temporal smoothing
%%methods, including using temporal smoothing, multi-language tag sets,
and machine learning to improve the accuracy of estimates. We found
that while the recall is relatively low due to the sparsity of photos
on any given day, the precision can be quite high, suggesting that
mining from photo sharing websites could be a reliable source of
observational data for ecological and other scientific research. In
future work, we plan to study additional features including using
more sophisticated computer vision techniques to analyze visual content. Also we plan to study a variety of other ecological phenomena,
including those for which high quality ground truth is not available,
such as migration patterns of wildlife and the distributions of
blooming flowers.
