\section{discussion and conclusion}


\hfill \break
\hfill \break
\subsection*{latest writing}
\hfill \break
\hfill \break

Big wrap-up, lots of ideas for future work.

\hfill \break
\hfill \break
===================================
\hfill \break
\hfill \break
\subsection*{workshop}

In this paper, we propose using photo-sharing social media sites as a
means of observing the state of the natural world, by automatically
recognizing specific types of scenes and objects in large-scale social
image collections. This work is an initial step towards a long-term
goal of monitoring important ecological events and trends through
online social media.  Our study shows that snowy scene recognition is
not nearly as easy a problem as one might expect, when applied to realistic
consumer images; our best result using
modern vision techniques gives 81\% accuracy. Nevertheless, as a proof-of-concept
we demonstrated that this recognition accuracy still yields a reasonable
map that approximates observations from satellites.
%However, using the current modern vision techniques to solve
%this problem, we are able to recreate the satellite map.  
We also test recognition algorithms on their ability to recognize a
particular species of flower, the California Poppy. In future work, we
plan to combine evidence from tags and other metadata with visual
features for more accurate estimates, and to develop novel techniques
for these challenging recognition problems. More generally, we hope
the idea of observing nature through photo-sharing websites will help
spark renewed interest in recognizing natural and ecological phenomenon in
consumer images.
%=======
%In this paper, we   propose using massive amount of latent visual information uploaded to social media as source  observing the state of the natural world by recognizing specific types of scenes and objects in large-scale social image collections. This work can be considered as  preliminary step towards long-term goal to understand the ecology phenomena using online social media.  Our study shows that snowy scene recognition  is not an easy problem as been expected. Best  results obtained using modern vision techniques is around 81\% accuracy which is not  high accuracy. However, using  the current modern vision techniques to solve this problem, we are able to recreate the satellite map.  We also  show the   current  recognition algorithms able to  detect a particular species of flower, the California
%Poppy at accuracy 72\%. 
%In future work we plan to find more sophisticated computer vision techniques for  these problems. Also, we plan to combine this work  with  the work of Zhang \textit{et al}~\cite{ecology2012www} which is based on textual information to improve the results. Also, we plan to study more ecological phenomena like migration patterns of wildlife.
%>>>>>>> .r2693

%% In future work we plan to find more sophisticated computer vision
%% techniques for these problems. Also, we plan t combine this work based
%% on visual information with the work of Zhang \textit{et
%%   al}~\cite{ecology2012www} which is based on textual information to
%% improve the results. Also, we plan to study more ecological phenomena
%% like migration patterns of wildlife.

\hfill \break
\hfill \break
===================================
\hfill \break
\hfill \break
\subsection*{www}


In this paper, we propose using the massive collections of
user-generated photos uploaded to social sharing websites as a source
of observational evidence about the world, and in particular as a way
of estimating the presence of ecological phenomena. As a first step
towards this long-term goal, we used a collection of 150 million
geo-tagged, timestamped photos from Flickr to estimate snow 
cover and greenery, and compared these estimates to fine-grained
ground truth collected by earth-observing satellites and ground stations. We compared
several techniques for performing the estimation from noisy, biased
data, including simple voting mechanisms and a Bayesian likelihood
ratio. We also tested several possible improvements to these basic
%%hp cr: removed "multi-language tag sets" as we did not report it in
%%this version
methods, including using temporal smoothing
%%methods, including using temporal smoothing, multi-language tag sets,
and machine learning to improve the accuracy of estimates. We found
that while the recall is relatively low due to the sparsity of photos
on any given day, the precision can be quite high, suggesting that
mining from photo sharing websites could be a reliable source of
observational data for ecological and other scientific research. In
future work, we plan to study additional features including using
more sophisticated computer vision techniques to analyze visual content. Also we plan to study a variety of other ecological phenomena,
including those for which high quality ground truth is not available,
such as migration patterns of wildlife and the distributions of
blooming flowers.


