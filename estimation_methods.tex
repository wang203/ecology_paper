\subsection{Estimation techniques}
\label{sec:methods}

Our goal is to estimate the presence or absence of a given ecological
phenomenon (like a species of plant or flower, or a meteorological
feature like snow) on a given day and at a given place,
using only the geo-tagged, time-stamped photos from Flickr. One way of viewing
this problem is that every time a user takes a photo of a phenomenon
of interest, they are casting a ``vote''  that the
phenomenon actually occurred in a given geospatial region. 
 We could
simply look for tags indicating the presence of a feature --
i.e. count the number of photos with the tag ``snow'' --  
but sources of noise and bias make this task 
challenging, including:
\begin{packed_itemize}
\item[---] \textit{Sparse sampling:} The geospatial distribution of photos
  is highly non-uniform. A lack of photos
  of a phenomenon in a region does not
  necessarily mean that it was not there. 
\item[---] \textit{Observer bias:} Social media users are younger and
  wealthier than average, and most live in North
  America and Europe.
\item[---] \textit{Incorrect, incomplete and misleading tags:}
  Photographers may use incorrect or ambiguous tags  ---
  e.g. the tag ``snow'' may refer to a snowy owl or interference on a
  TV screen.
\item[---] \textit{Measurement errors:} Geo-tags and timestamps are
  often incorrect (e.g. because people   forget to set their camera clocks).
\end{packed_itemize}

\xhdr{A statistical test.}  We introduce a simple probabilistic model
and use it to derive a statistical test that can deal with some such
sources of noise and bias. The test could be used for estimating the
presence of any phenomenon of interest; without loss of generality we
use the particular case of snow here, for ease of explanation.  Any
given photo either contains evidence of snow (event $s$) or does not
contain evidence of snow (event $\bar{s}$).  We assume that a given
photo taken at a time and place with snow has a fixed probability $P(s
| snow)$ of containing evidence of snow; this probability is less than
1.0 because many photos are taken indoors, and outdoor photos might be
composed in such a way that no snow is visible. We also assume that
photos taken at a time and place without snow have some non-zero
probability $P(s | \overline{snow})$ of containing evidence of snow;
this incorporates various scenarios including incorrect timestamps or
geo-tags and misleading visual evidence (e.g.  man-made
snow).

Let $m$ be the number of snow
photos (event $s$), and $n$ be the number of non-snow photos (event
$\bar{s}$) taken at a place and time of interest. Assuming that each photo is captured
independently, we can use Bayes' Law to
derive the probability that a given place has snow
given its number of snow and non-snow photos,
%
%\newcommand{\smsn}{\overbrace{s\cdots s}^{m},\overbrace{\overline{s}\cdots \overline{s}}^{n}}
%%hp cr: \bar{s}^n
\newcommand{\smsn}{s^m, \bar{s}^n}
\newcommand{\smsntwo}{s^m, \bar{s}^n}
%%\newcommand{\smsn}{s^m, s^n}
%%\newcommand{\smsntwo}{s^m, s^n}
%\newcommand{\smsntwo}{\underbrace{s\cdots s}_{m},\underbrace{\overline{s}\cdots \overline{s}}_{n}}
\begin{eqnarray*}
P(snow|\smsn)  &=&\frac{ P(\smsn|snow)P(snow)}{P(\smsntwo)}  \\
&=&\frac{{m+n\choose m}p^{m}(1-p)^{n}P(snow)}{P(\smsntwo)},  
\end{eqnarray*}
%
where we write $s^m, \bar{s}^n$ to denote $m$ occurrences of event $s$ and $n$ occurrences of event $\bar{s}$, and where $p=P(s|snow)$ and $P(snow)$ is the prior probability of snow. A similar derivation gives the posterior probability that the bin does not contain snow,
%
\begin{eqnarray*}
P(\overline{snow}|\smsn)  &=&\frac{{m+n\choose m}q^{m}(1-q)^{n}P(\overline{snow})}{P(\smsntwo)},  
\end{eqnarray*}
%
where $q=P(s|\overline{snow})$. 
%
Taking the ratio between these two posterior probabilities yields a likelihood ratio,
%
\begin{eqnarray}
\frac{P(snow|\smsn)}{P(\overline{snow}|\smsntwo)}
%\\=\frac{\frac{{m+n\choose m}p^{m}(1-p)^{n}P(snow)}{P(\overbrace{s\cdots s}^{m},\overbrace{\overline{s}\cdots \overline{s}}^{n})}}{\frac{{m+n\choose m}q^{m}(1-q)^{n}P(\overline{snow})}{P(\overbrace{s\cdots s}^{m},\overbrace{\overline{s}\cdots \overline{s}}^{n})}}
&=&\frac{P(snow)}{P(\overline{snow})}\left(\frac{p}{q}\right)^{m}\left(\frac{1-p}{1-q}\right)^n.
\label{eq:conf}
\end{eqnarray}
%
This ratio can be thought of as a measure of the confidence that a
given time and place actually had snow, given photos from Flickr.

A simple way of classifying a photo into a positive event $s$ or a
negative event $\bar{s}$ is to use text tags. We identify a
set ${\cal S}$ of tags related to a phenomenon of
interest. Any photo tagged with at least one tag in ${\cal S}$ is
declared to be a positive event $s$, and otherwise it is considered a
negative event $\bar{s}$. For the snow detection task, we use the set
${\cal S}$=\{snow, snowy, snowing, snowstorm\}, which we selected
by hand.

%%, which we chose by
%%looking at the 200 most frequent Flickr tags and hand selecting those
%%directly relevant to snowfall.

The above derivation assumes that photos are taken independently of
one another, which is generally not true in reality. One particular
source of dependency is that photos from the same user are highly
correlated with one another.  To mitigate this problem, instead of
counting $m$ and $n$ as numbers of \textit{photos}, we instead let $m$ be 
the number of \textit{photographers} having at least one photo with evidence of snow,
while $n$ is the numbers of photographers who did not upload any
photos with evidence of snow.

The probability parameters in the likelihood ratio of
equation~(\ref{eq:conf}) can be directly estimated from training data
and ground truth. For example, for the snow cover results
presented in Section~\ref{sec:results}, the learned parameters are: $p
= p(s|snow) = 17.12\%$, $q = p(s|\overline{snow}) = 0.14\%$.  In other
words, almost 1 of 5 people at a snowy place take a photo containing
snow, whereas about 1 in 700 people take a photo containing evidence
of snow at a non-snowy place.

Figure~\ref{fig:samplemap} shows a visualization of the likelihood
ratio values for the U.S. on one particular day using this simple
technique with ${\cal S}$=\{snow, snowy, snowing, snowstorm\}.  High
likelihood ratio values are plotted
in green, indicating a high confidence of snow in a geospatial bin,
while low values are shown in blue and indicate high confidence of 
no snow.  Black areas indicate a likelihood ratio 
near 1, showing little conference either way, and grey areas lack
data entirely (having no Flickr photos in that bin on that day).
